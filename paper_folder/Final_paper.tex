\documentclass{article}

% if you need to pass options to natbib, use, e.g.:
%     \PassOptionsToPackage{numbers, compress}{natbib}
% before loading neurips_2020

% ready for submission
% \usepackage{neurips_2020}

% to compile a preprint version, e.g., for submission to arXiv, add add the
% [preprint] option:
\usepackage[preprint]{neurips_2020}

% to compile a camera-ready version, add the [final] option, e.g.:
%     \usepackage[final]{neurips_2020}

% to avoid loading the natbib package, add option nonatbib:
%    \usepackage[nonatbib]{neurips_2020}

\usepackage[utf8]{inputenc} % allow utf-8 input
\usepackage[T1]{fontenc}    % use 8-bit T1 fonts
\usepackage{hyperref}       % hyperlinks
\usepackage{url}            % simple URL typesetting
\usepackage{booktabs}       % professional-quality tables
\usepackage{amsfonts}       % blackboard math symbols
\usepackage{nicefrac}       % compact symbols for 1/2, etc.
\usepackage{microtype}      % microtypography
\usepackage{graphicx}
\usepackage{float}
\restylefloat{table}
\usepackage{array}
\newcolumntype{L}{>{\centering\arraybackslash}m{3cm}}

\title{Formatting Instructions For NeurIPS 2020}

% The \author macro works with any number of authors. There are two commands
% used to separate the names and addresses of multiple authors: \And and \AND.
%
% Using \And between authors leaves it to LaTeX to determine where to break the
% lines. Using \AND forces a line break at that point. So, if LaTeX puts 3 of 4
% authors names on the first line, and the last on the second line, try using
% \AND instead of \And before the third author name.

\title{Title\\ BLABLABLA
	\large Advanced Machine Learning}

\author{Caitlin Loftus \\
	\texttt{cloftus@uchicago.edu}  \\
	The University of Chicago
	\AND
	Roberto Barroso-Luque\\
	\texttt{barrosoluquer@uchicago.edu} \\
    The University of Chicago\\
	\AND
	Rukhshan Mian\\
	\texttt{rukhshan@uchicago.edu} \\
	The University of Chicago\\}

\begin{document}
\maketitle

\begin{abstract}{
		Our study blabla
	}
\end{abstract}

\newpage
\section{Introduction and Background}{
BLABLABLA

}
\newpage

\section{Methodology}{BLABLABLA
}
\newpage

\section{Results}{	BLABLABLA
}
\newpage
\section{Conclusion, further research and limitations}{
BLABLABLA
}

\section{References}\label{sec_ref}

[1] Grimmer J. \& Stewart B.M (2013): Text as Data: The Promise and Pitfalls of Automatic Content Analysis Methods for Political Texts. 
{\it Political Analysis}

[2] Rule A. , Cointet J.P. \& Bearman P.S. (2015). Lexical shifts, substantive changes, and continuity in State of the Union discourse, 1790–2014 
{\it PNAS}

[3]Savoy J. (2018): Analysis of the style and the rhetoric of the 2016 US presidential primaries \\ 
{\it Digital Scholarship in the Humanities}

[4]Stuckey M. (2010): Rethinking the Rhetorical Presidency and Presidential Rhetoric \\ 
{\it Communication Arts and Sciences}
\newpage
\pagebreak

\end{document}
